\documentclass{ltjsarticle}
\begin{document}

\title{プログラミング技法2_課題1レポート}
\author{Sakata Seinosuke}
\maketitle

\section*{課題1-1}
\subsection*{ソースコード}
課題1-1のソースコードをlist1に示す。

\\サンプルプログラムからの変更点は
\begin{itemize}
    \item 入力部分のintをfloatに
    \item 初期値を1に
\end{itemize}
である。

\subsection*{出力結果}
課題1-1の出力結果を
\\一回目の実行では5を入力した結果、二回目の実行では0.25を入力した結果を示している。

どちらも正しく出力されていることがわかる。

\subsection*{工夫した箇所}
平方根は1より大きければ元の数より小さく、1より小さければ元の数より大きくなるという性質がある。そのため初期値を1とすることで、入力された数に応じて数を増減させるだけで目的の値を得ることができる。

\section*{課題1-2}
\subsection*{ソースコード}
あたり前過ぎて気に止めることもないですが、きとんと改行命令を出していなくても自動で改行します。
\\あ、ギリ改行されなかったか。

\subsection*{出力結果}
あたり前過ぎて気に止めることもないですが、きとんと改行命令を出していなくても自動で改行します。
\\あ、ギリ改行されなかったか。

\subsection*{工夫した箇所}
あたり前過ぎて気に止めることもないですが、きとんと改行命令を出していなくても自動で改行します。
\\あ、ギリ改行されなかったか。

\section*{課題1-3}
\subsection*{ソースコード}
あたり前過ぎて気に止めることもないですが、きとんと改行命令を出していなくても自動で改行します。
\\あ、ギリ改行されなかったか。

\subsection*{出力結果}
あたり前過ぎて気に止めることもないですが、きとんと改行命令を出していなくても自動で改行します。
\\あ、ギリ改行されなかったか。

\subsection*{工夫した箇所}
あたり前過ぎて気に止めることもないですが、きとんと改行命令を出していなくても自動で改行します。
\\あ、ギリ改行されなかったか。。

\section*{課題1-4}
\subsection*{ソースコード}
あたり前過ぎて気に止めることもないですが、きとんと改行命令を出していなくても自動で改行します。
\\あ、ギリ改行されなかったか。

\subsection*{出力結果}
あたり前過ぎて気に止めることもないですが、きとんと改行命令を出していなくても自動で改行します。
\\あ、ギリ改行されなかったか。

\subsection*{工夫した箇所}
あたり前過ぎて気に止めることもないですが、きとんと改行命令を出していなくても自動で改行します。
\\あ、ギリ改行されなかったか。
\end{document}